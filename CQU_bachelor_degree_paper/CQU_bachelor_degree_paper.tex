\documentclass[12pt, a4paper, oneside]{ctexart}
\usepackage{amsmath}  % 这是数学公式宏包
\usepackage{geometry}  % 页边距宏包
\usepackage{fontspec}  % 设置字体
\usepackage{titlesec}  % 设置标题字体
\usepackage{titletoc}  % 目录格式
\usepackage{fancyhdr}  % 页眉页脚设置
\usepackage{graphicx}  % 图片
\usepackage{caption}  % 图片标题设置
\usepackage{multirow}  % 表格单元格纵向合并
\usepackage{tabularx}
\usepackage{setspace}

\newCJKfontfamily\SimHei{SimHei}  % 设置黑体
\newCJKfontfamily\SimSun{SimSun}  % 设置宋体

\geometry{top=30mm, left=35mm, right=25mm, bottom=25mm, headheight=16mm, footskip=10mm}  % 页边距设置
\setmainfont{Times New Roman}
\setCJKmainfont{SimSun}
\titleformat{\section}[block]{\SimHei\centering\fontsize{16pt}{20pt}}{\textbf{\thesection}}{0em}{\vspace{-5pt}}  % 16磅字体大小(三号),20pt行距
\titleformat{\subsection}[block]{\SimHei\fontsize{15pt}{20pt}}{\textbf{\thesubsection}}{0em}{\vspace{-5pt}}
\titleformat{\subsubsection}[block]{\SimHei\fontsize{14pt}{20pt}}{\textbf{\thesubsubsection}}{0em}{\vspace{-5pt}}
% 三号对应16pt,小三15pt,四号14pt,小四12pt

% 目录格式
\titlecontents{section}[0.5cm]{\SimSun\fontsize{12pt}{20pt}}{\contentslabel{1em}}{}{\titlerule*[0.5pc]{$\cdot$}\contentspage}
\titlecontents{subsection}[1.5cm]{\SimSun\fontsize{12pt}{20pt}}{\contentslabel{1.8em}}{}{\titlerule*[0.5pc]{$\cdot$}\contentspage}
\titlecontents{subsubsection}[2.5cm]{\SimSun\fontsize{12pt}{20pt}}{\contentslabel{2.6em}}{}{\titlerule*[0.5pc]{$\cdot$}\contentspage}

\pagestyle{fancy}
\fancyhf{}
\fancyhead[L]{\SimSun\fontsize{10.5pt}{12.6pt} 这是论文题目}  % 左页眉
\fancyhead[R]{\SimSun\fontsize{10.5pt}{12.6pt} \rightmark}  % 右页眉
\renewcommand{\sectionmark}[1]{\markright{#1}}  % 每个section更新rightmark
\fancyfoot[C]{\fontsize{10.5pt}{12.6pt} \thepage}  % 页码设置

\renewcommand{\thefigure}{\arabic{section}.\arabic{figure}}  % 图片编号设置
\renewcommand{\thetable}{\arabic{section}.\arabic{table}}  % 表格编号设置
\captionsetup[figure]{font={small,stretch=2}}  % 五号,行距20pt
\captionsetup[table]{font={small,stretch=2}}  % 五号,行距20pt
\captionsetup[table]{justification=raggedright,singlelinecheck=false}  % 表格标题左对齐
\setlength{\abovecaptionskip}{0pt}  % 图表标题与图标间距
\setlength{\belowcaptionskip}{0pt}
\newcolumntype{C}{>{\centering\arraybackslash}X} % 自定义居中对齐的列类型C
\renewcommand{\arraystretch}{1.2}  % 表格的单元格高度
\captionsetup[figure]{labelsep=space}  % 标题标号和标题文本间的连接符
\captionsetup[table]{labelsep=space}  % 标题标号和标题文本间的连接符

%\setlength{\parskip}{0pt}  % 段间距为0

\title{重庆大学本科毕业论文\LaTeX 模板}
\author{Jacken Wu}
\date{\today}

\begin{document}

\maketitle
\thispagestyle{empty}

\newpage

\section*{摘要}
\markright{摘要}
\addcontentsline{toc}{section}{摘要}
\setcounter{page}{1}
\pagenumbering{Roman}  % 页码改为罗马数字

\newpage

\addcontentsline{toc}{section}{目录}
\SimSun{\tableofcontents}

\newpage

\section{这是第一章}
\setcounter{page}{1}
\pagenumbering{arabic}  % 页面改为阿拉伯数字

\subsection{这是第一小节}

\subsubsection{这是小小节}

这是正文。这是正文。这是正文。这是正文。这是正文。这是正文。这是正文。这是正文。这是正文。这是正文。这是正文。这是正文。这是正文。这是正文。这是正文。这是正文。这是正文。这是正文。这是正文。这是正文。这是正文。这是正文。这是正文。这是正文。这是正文。这是正文。这是正文。这是正文。这是正文。这是正文。这是正文。这是正文。这是正文。这是正文。这是正文。这是正文。这是正文。这是正文。这是正文。这是正文。这是正文。这是正文。这是正文。这是正文。这是正文。这是正文。这是正文。这是正文。这是正文。这是正文。这是正文。这是正文。这是正文。这是正文。这是正文。这是正文。这是正文。这是正文。这是正文。这是正文。这是正文。这是正文。这是正文。这是正文。

这是正文。这是正文。这是正文。这是正文。这是正文。这是正文。这是正文。这是正文。这是正文。这是正文。这是正文。这是正文。这是正文。这是正文。这是正文。这是正文。这是正文。这是正文。这是正文。这是正文。这是正文。这是正文。这是正文。这是正文。这是正文。这是正文。这是正文。这是正文。这是正文。这是正文。这是正文。这是正文。这是正文。这是正文。这是正文。这是正文。这是正文。这是正文。这是正文。这是正文。这是正文。这是正文。这是正文。这是正文。这是正文。这是正文。

\subsubsection{这也是小小节}

正文。

\begin{table}[htb]
	\centering
	\caption{这是一个表格}
	\begin{tabularx}{\textwidth}{|C|C|C|}  % c居中,l左对齐,X填充,C自定义的居中填充
		\hline
		第一列 & 第二列 & 第三列 \\
		\hline
		\multicolumn{2}{|c|}{12} & 3 \\
		\hline
		\multirow{2}*{11} & 2 & 3 \\
		\cline{2-3}
		~ & 2 & 3 \\
		\hline
		1 & 2 & 3 \\
		\hline
	\end{tabularx}
\end{table}

\subsection{这是第二小节}

这是一个公式:

$$ z = \dfrac{x^2}{3 y_1^2 y_2^2} \eqno{(1.1)} $$

\newpage

\section{这是第二章}

\subsection{程序员的自我修养}

\begin{figure}[htb]
	\centering
	\includegraphics[width=0.6\textwidth]{example-image}
	\caption{这是图片的标题}
	\label{2.1}
\end{figure}

图2.1为一个示例。

\subsection{程序员的脱发治疗}

没救了。

\end{document}
